% %%%%%%%%%%%%%%%%%%%%%%%%%%%%%%%%%%%%%%%%%%%%%%%%%%%%%%%%%%%%%%%%%%%%%%%%%%%%%%%%
%
% example.tex :
% Based on example.tex by J. Dreissig (julian.dreissig@physik.de) for GSI Summer 
% Student Program 1999.
% Recycled for the Summer Student Program in 2007 and 2008.
% 
% This Latex2e file contains an example layout for your final report.
%
%   2007:
%   Maciek Sobczak, macieksobczak@wp.pl
%   Friedemann Zenke, fzenke@gmail.com
%
%   2008:
%   04herbst@edu.uni-graz.at
%
% %%%%%%%%%%%%%%%%%%%%%%%%%%%%%%%%%%%%%%%%%%%%%%%%%%%%%%%%%%%%%%%%%%%%%%%%%%%%%%%%
% DO NOT EDIT!
% BEGIN Header
\documentclass[twocolumn,gsifonts,twoside]{gsipaper}
\usepackage{a4wide,gsiindex,helvet}
\usepackage[english]{babel}
\usepackage{amsmath,amsfonts,amssymb,verbatim,float}
\usepackage[pdftex]{graphicx} % Graphic files: pdf or jpg

\begin{document}
% END Header

% Edit below this line
% %%%%%%%%%%%%%%%%%%%%%%%%%%%%%%%%%%%%%%%%%%%%%%%%%%%%%%%%%%%%%%%%%%%%%%%%%%%%%%%%


% -= NOTA BENE =-
% Replace the title, abstract, authors, and addresses within the curly brackets
% with your own title, authors, and  addresses. You may have as many authors and
% addresses as you need.  The control sequences of author and addresses can be
% repeated as often as necessary. Footnotes in titles, adresses and the abstract
% need a special procedure and are described below.

\title{Quality Assurance of silicon sensors for CBM-STS}

\abstract{This report introduces CBM-STS detector and the Quality Test Center(QTC)
	in GSI. It also details the hardware and setup for quality assurance as well as
	the QA procedures, which includes optical and electrical inspection. The database
	framework for CBM-STS QA has been implemented using	the FairDB SQL Interface, 
	and QA data schemes would be shown in this paper. In addition, several diagrams
	extracting from measurement data are given.
	}

\shortauthor{Wu, Yitao}
\shorttitle{CBM-STS QA}

\author{Yitao Wu}            % first author
\address{University of Science and Technology of China, torrence@mail.ustc.edu.cn}

%\author{next author}                  % if you have more than one author,
%\address{next address}                % use these lines for the next one.

\maketitle
%\thispagestyle{empty}

% -= NOTA BENE =-
% Start your text with the following command. 

\section{Introduction}
The Compressed Baryonic Matter (CBM) experiment will carry out
systematic research on the properties of nuclear matter under extreme conditions,
in particular, at highest net baryon densities. These conditions will
be met by colliding beams of heavy ions on targets in the energy range from
2 to 14, eventually 45 GeV/nucleon, as they will be provided with highest
intensities by the heavy-ion synchrotron SIS-100, and in a future stage
by the SIS-300 machine of the Facility for Anti-proton and Ion Research
(FAIR) at GSI, Darmstadt, Germany.

\subsection{Silicon Tracking System}
Objective, Requirements, Design, Principle, Geometry, Readout, DAQ

The Silicon Tracking System (STS) is the central detector fo the CBM
experiment at FAIR. Its task is the standalone trajectory reconstruction
of the high multiplicities of charged particles originating from high-rate
beam-target interactions. The silicon microstrip detectors must be
radiation hard and are read out by a fast self triggering front-end
electronics.

\subsection{Quality Test Center}
Motivation program of QA. Task and condition in QTC

Due to a large-scale sensor production, the effective QA of sensors can be
achieved by a collective effort among the Quality Test Centers. [Figure]
indicates the variety of quality assurance tests and the corresponding QTC
in charge. The overall QA program can be represented as a combination of
tests performed on sensor and strip level.

\paragraph{Optical inspection} is intended to identify any kinds of surface
defects. This test will be performed on 100\% of the sensors at the EKU
test center. Sensors with identified defects will undergo further electrical
tests.

\paragraph{Bulk electrical tests} are performed to determine the overall
sensor behavior at the operational conditions. Current-voltage (I-V) and
capacitance-voltage (C-V) tests are conducted after optical inspection. The
I-V tests determines the leakage current of a sensor while the C-V tests
determines the full depletion voltage of a sensor.

\paragraph{Long-term stability tests} will be performed on a fraction of the
sensors. The leakage current of the sensors will be monitored for a time period
of 48-72 hours in order to evaluate their stability.

\paragraph{Microscopic electrical tests} focus on determination of various sensor
parameters and their correspondence with specifications, mainly coupling
capacitance $C_{c}$, interstrip capacitance $C_{is}$, interstrip resistance
$R_{is}$, and bias resistance $R_{bias}$.

\paragraph{Strip diagnostic tests} aim to identify strip defects originating
from the complex fabrication process that cannot be identified by the optical
inspection. These defects comprise mainly ohmic contacts or short circuits
between the strip implant and the readout strip (so-called ``pinholes''), short
circuits between two or more readout strips, and strips exhibiting high leakage
current.

\section{Hardware and Setup}
\subsection{Testing conditions}
\subsection{Storage conditions}
\subsection{Equipment and testing tools/devices}

\section{Quality Assurance in GSI}
\subsection{Optical Inspection}
\subsection{Electrical Inspection}

\section{QA Database}
\subsection{FairDB}
\subsection{Some Results}

\section{Summary}

\section*{Acknowledgments}

Olga, Anton;
Muksym?, Hanna, Adrian, ...;
Summer students;
J{\"o}rn, Gabi, organizers, lecturers, GSI

\begin{thebibliography}{99}

\bibitem{ywu:pl}
P. Larionov, \textsl{Systematic Irradiation Studies and Quality Assurance of
Silicon Strip Sensors for the CBM Silicon Tracking System}, PhD thesis, Johann
Wolfgang Goethe-Universit{\"a}t in Frankfurt am Main, 2016.

%\cite{Heuser:2016coc}
\bibitem{ywu:jh} 
J.~M.~Heuser [CBM Collaboration], \textsl{Status of the Compressed Baryonic Matter
 Experiment at FAIR and Its Silicon Tracking System},
Acta Phys.\ Polon.\ Supp.\  {\bf 9}, 221 (2016).
%%CITATION = doi:10.5506/APhysPolBSupp.9.221;%%

\bibitem{ywu:ob}
O. Bertini etc., \textsl{Specifications for Quality Assurance of Microstrip
	Sensors in the CBM Silicon Tracking System}, STS Note 2016-1, 2016

\bibitem{ywu:fr}
F.~Uhlig etc., \textsl{The FairRoot framework},
J.\ Phys.\ Conf.\ Ser.\  {\bf 396}, 022001 (2012).
%%CITATION = doi:10.1088/1742-6596/396/2/022001;%%

\bibitem{ywu:db}
D. Bertini. \textsl{FairRoot Virtual Database (User Manual)}. 2013

\end{thebibliography}

% %%%%%%%%%%%%%%%%%%%%%%%%%%%%%%%%%%%%%%%%%%%%%%%%%%%%%%%%%%%%%%%%%%%%%%%%%%%%%%%%
% Do NOT edit below this line
% %%%%%%%%%%%%%%%%%%%%%%%%%%%%%%%%%%%%%%%%%%%%%%%%%%%%%%%%%%%%%%%%%%%%%%%%%%%%%%%%

\end{document}

% %%%%%%%%%%%%% END OF example.tex %%%%%%%%%%%%%%%%%%%%%%%%%%%%%%%%%%%%%%%%%%%%%%%